\section{Introduction}
\label{sec:Introduction}

\jel{} is a program animation system intended for teaching introductory
programming. Programs are animated fully or semi-automatically, requiring
no modifications or annotations on the part of the instructor or student.
While this limits the flexibility of the animation, \jel{} is extremely
simple to use so that it is easily accepted by true novices, as well as
by their teachers who do not have to invest in learning how to prepare
animations.

\jel{} is written in Java for portability and animates program that are
written in Java. \jel{} uses modified version of \djava{}
(section~\ref{sec:DynamicJava}) as a front-end and
modified version of \jel{}~2000's animation engine
(section~\ref{sec:Visualization_Engine}) as its back-end. The user interface
was also adopted from \jel{}~2000 (section~\ref{sec:User_Interface}).
For an overview of the design see section~\ref{sec:Concept_Design}.

\djava{} is a Java source code interpreter written in Java. This
application is open source and can be freely obtained from
\url{http://www.koala.ilog.fr/djava/}. At the moment, DynamicJava is
almost fully compliant with Java language specifications and supports
multi-threading.

To make these two separate systems communicate (section~\ref{sec:Communications_Model})
with each other a new intermediate language was designed (section~\ref {sec:Intermediate_Code}
and Intermediate Language Specification). In this way the system should be
more flexible to new extensions (section~\ref{sec:Extending_Jeliot_System}).

\subsection{History}

The first member of the \jel{} family of program animation systems was
called Eliot; it was developed by Erkki Sutinen and Jorma Tarhio and
their students at the University of Helsinki. Eliot was written in C
and used the Polka animation library. Later a version was written in
Java and the name was changed to \jel{}. (This version is now called
\jel{}~I to distinguish it from later versions.)

\jel{}~I was a flexible system: the user could choose the variables to
be animated, the graphical form of the animated elements and multiple
views of the same program. It proved too difficult for teaching novices,
so a new version called \jel{}~2000 was developed at the Weizmann Institute
of Science by Pekka Uronen under the supervision of Moti Ben-Ari.
A pedagogical experiment carried out by Ronit Ben-Bassat Levy proved
the effectiveness of \jel{} in teaching introductory programming.

\jel{}~2000 only implemented a very small subset of the Java language.
The current version \jel{}~3 is a re-implementation designed to
significantly extend the range of Java features that are animated,
in particular, to cover object-oriented programming. The design and
implementation was carried out at the University of Joensuu by
Niko Myller and Andr{\'{e}} Moreno-Garc{\'{\i}}a, under the supervision
of Moti Ben-Ari and Erkki Sutinen. For an extensive discussion of the
history of \jel{}, see \citep{Benari2002a}.

\subsection{Release Notes}
\label{sec:Release_Notes}

{\bf version 3.0}

{\bf version 3.01}

{\bf version 3.1}

{\bf version 3.2}
