\section{Managing \jel{} System}
\label{sec:Managing_Jeliot_3_System}

In this chapter we explain what are the system requirements of \jel{},
what files and directories different distributions contain and
how the source code distribution can be build. This will lead us to
the next chapter containing information about how to extend \jel{} system.

\subsection{System Requirements}
\label{sec:System_Requirements}

\subsection{End User Distribution}
\label{sec:End_User_Distribution}

\subsection{Source Code Distribution}
\label{sec:Source_Code_Distribution}

\subsubsection{Directory Hierarchy}

\jel{} is available as zip file containing all the sources
and files needed to build it. After unzipping the file we will
find the following directories:
\begin{description}
\item[\p{lib}] Contains the tools needed for the automated build process.
\item[\p{resources}] Contains information messages used by DynamicJava
to alert from errors.
\item[\p{src}] Contains the source code for \jel{}. It is divided into
the following subfolders.
\begin{description}
\item[\p{docs}] Contains the web pages that are used for the help page
and the about page. It also contains the licenses under which
\jel{} is distributed.
\item[\p{examples}] Contains the examples that will be available for users
of \jel{}, new examples can be added here.
\item[\p{images}] Those images used in the user interface of \jel{}.
\item[\p{jeliot}] All source files related to \jel{} visualization engine (\p{theatre}),
m-code interpretation (\p{ecode}) and the graphical user interface (\p{gui}).
\item[\p{koala}] The modified source code of \djava{}.
\end{description}
\end{description}

\subsubsection{How to Build Jeliot 3 From the Sources}

Being used to the build tool used by DynamicJava (Ant), it was
modified to build also \jel{}. Ant is a UNIX make-clone oriented
to build Java programs based on an XML configuration file \url{http://ant.apache.org}.
When unzipping the source code, three files will appear on the
destination directory:
\begin{description}
\item[build.xml] This file defines the possible targets and its tasks
that we want to perform with Ant. It includes some properties
to customize the output. For example, you can rename the minor
version of \jel{} by modyfing the property named minor. For adding
more targets you should refer to Ant manual page
(\url{http://ant.apache.com/manual/index.html}).
\item[build.bat and build.sh] These are the batch files that
will invoke Ant with one of the arguments (targets) that you
can provide to it:
\begin{description}
\item[compile] \jel{} source code will be compiled and classes obtained
will be located at classes subfolder. To run \jel{} from this
point you should enter the command java jeliot.Jeliot inside
the classes subfolder.
\item[dist] This argument will compile (if necessary), create the
jar files and zip them. This way we will obtain a binary zip
file of \jel{} Jeliot3\$\{minor\}.zip and another
zip file of \jel{} source code Jeliot3\$\{minor\}-src.zip.
These files are ready to be distributed without any modifications.
\item[clean] deletes every file created by the build tool.
\end{description}
\end{description}
Notice that you must set JAVA\_HOME to point at your Java Development
Kit installation.

A normal session could consist of the following commands:
\begin{verbatim}
c:\jeliot3> build compile
\end{verbatim}
The compiled class files are now in the folder classes.
\begin{verbatim}
c:\jeliot3> cd classes
c:\jeliot3\classes> java jeliot.Jeliot
\end{verbatim}
We can run Jeliot with the command above. But if we want to make a
distribution we need to do as follows and we will get two separate
distributions one with the compiled files and one with the source codes.
\begin{verbatim}
c:\jeliot3\classes> cd ..
c:\jeliot3> build dist
c:\jeliot3> cd Jeliot3
c:\jeliot3\Jeliot3> jeliot
\end{verbatim}

